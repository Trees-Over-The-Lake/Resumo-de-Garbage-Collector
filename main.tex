\documentclass{article}

\title{Resumo sobre o Funcionamento dos \emph{Garbage Collectors} de Java e Golang}
\author{Lucas Santiago \and Rafael Amauri \and Tallys Assis}
\date{18 de dezembro de 2023}

\usepackage{indentfirst}
\usepackage[utf8]{inputenc}
\usepackage[brazil]{babel}

\begin{document}
    \maketitle

    \section*{Introdução}

    Um coletor de lixo, no contexto da programação de computadores, refere-se a um componente essencial nos sistemas que gerenciam a
    alocação e desalocação de memória. Sua função principal é identificar e recuperar automaticamente os blocos de memória que
    não estão mais em uso, liberando recursos e prevenindo vazamentos de memória. Essa tarefa é particularmente crucial em
    linguagens de programação que não oferecem suporte explícito para gerenciamento de memória, como é o caso de linguagens
    baseadas em Java, Python e Go. Ao automatizar a coleta e remoção de objetos não utilizados, os coletores de lixo contribuem
    para a eficiência do sistema, melhorando a estabilidade, prevenindo falhas devido à falta de memória e facilitando o desenvolvimento
    de software ao aliviar os programadores da responsabilidade manual de gerenciamento de memória.

    A grande vantagem de um garbage collector é tirar o peso do gerenciamento de memória das costas do programador,
    o que pode se mostrar extremamente importante na parte de produtividade, sendo que gerenciamento de memória de forma bem-feita
    se torna uma tarefa altamente laboriosa quando se trata de programas grandes.

    \subsection*{\emph{Reaching Definition}}
    \emph{Reaching definitions} se refere a uma parte de um conjunto de técnicas mais amplo chamado ``análise de fluxo de dados",
    que serve para a otimização em compiladores. A análise de \emph{reaching definitions} é mais relevante na fase de otimização
    de compiladores, onde é usada para identificar os pontos em um programa onde uma variável pode ser definida antes de ser usada.
    Essa informação ajuda os compiladores a realizar otimizações, como eliminação de código morto ou movimentação de código para otimizar o desempenho.

    Por outro lado, no contexto de \emph{garbage collection}, o que é crucial é a análise de tempo de vida (\emph{lifetime analysis}) de objetos.
    O \emph{garbage collector} precisa identificar quando um objeto não é mais acessível a partir do programa principal, indicando que ele pode ser coletado.
    Considere o código abaixo por exemplo:

    \begin{verbatim}
        int x
        /* várias contas que não utilizam x */
        foo(x).
    \end{verbatim}

    Uma otimização que pode ser feita aqui é a movimentação da declaração de $x$ para logo acima da função $foo$, pois assim estamos
    otimizando o uso de memória ao não alocar espaço para $x$ antes de ser necessário. Essa otimização também consegue tirar vantagem
    de acessar a variável em cache ao invés de RAM, pois logo após a declaração, é possível que $x$ se encontre armazenado na
    cache do sistema, assim garantindo um ganho de performance por tempo de acesso mais rápido.

    Considere o próximo código:

    \begin{verbatim}
        int x
        /* várias contas que não utilizam x */
        return 0
    \end{verbatim}

    Nesse código $x$ nunca é usado, sendo a declaração de $x$ considerada uma seção de código morto.
    Nesse caso, uma otimização que \emph{reaching definitions} faz é a eliminação da declaração de $x$ para não gastar memória
    de forma desnecessária no programa.

    A análise de \textit{Reaching Definitions} é crucial para várias otimizações de compilador e para a
    análise estática de código. Ela permite ao compilador:

    \begin{itemize}
        \item Identificar e eliminar código morto.
        \item Realizar a propagação de constantes.
        \item Otimizar o gerenciamento de variáveis em memória.
        \item Melhorar a eficiência na execução de programas.
    \end{itemize}

    Considere o seguinte trecho de código:

    \begin{verbatim}
        x = 10; // Definição A
        y = x + 1; // Definição B
        x = 20; // Definição C
        z = x + 2; // Definição D
    \end{verbatim}

    Aqui, a {Definição A} de {x} alcança a {Definição B}, mas não a {Definição D},
    pois {x} é redefinido em {Definição C}. Com isso. é possível notar que a análise de
    \textit{Reaching Definitions} é um componente essencial na otimização de compiladores,
    fornecendo informações valiosas sobre o uso de variáveis e ajudando a melhorar a
    eficiência do código gerado.

    \subsection*{\emph{Lifetime Analysis}}

    A análise de tempo de vida (\textit{lifetime analysis}) em compiladores é um processo crucial que determina o
    período durante o qual as variáveis e objetos em um programa são válidos e podem ser acessados. Esta análise é
    essencial para a otimização de memória e prevenção de erros comuns relacionados à gestão de recursos.

    Durante a compilação, o compilador realiza uma análise estática do código para identificar o ciclo de vida
    de cada variável. O \textit{lifetime} de uma variável começa quando é criada (por exemplo, através de uma declaração)
    e termina quando não é mais acessível no código. Esta análise é fundamental para
    alocar e desalocar memória de forma eficiente, evitar vazamentos de memória, acessos a memória não
    inicializada ou liberada e otimizar o uso de registradores e a pilha de execução.

    Existem diferentes métodos para realizar a análise de tempo de vida, dependendo da linguagem de programação e da complexidade do compilador:

    \begin{itemize}
        \item \textbf{Análise Estática}: Baseia-se no código-fonte sem considerar a execução dinâmica.
        Utilizada em linguagens como GO.

        \item \textbf{Análise Baseada em Escopo}: Determina o tempo de vida de variáveis com base em
        seu escopo (por exemplo, variáveis locais).

        \item \textbf{Análise de Fluxo de Dados}: Usa o grafo de fluxo de controle para entender
        como as variáveis são usadas e modificadas ao longo do programa.
    \end{itemize}

    A análise de tempo de vida é crucial para a otimização de compiladores, pois permite:

    \begin{itemize}
        \item Gerenciar eficientemente os recursos de memória.

        \item Implementar técnicas de otimização, como realocação de registradores e eliminação de subexpressões comuns.

        \item Garantir a segurança do programa, prevenindo erros relacionados à memória.
    \end{itemize}

    A análise de tempo de vida durante a compilação é um aspecto fundamental na engenharia de software,
    impactando diretamente a eficiência, segurança e confiabilidade dos programas. Ao entender e
    otimizar o ciclo de vida das variáveis, os compiladores desempenham um papel vital na gestão
    de recursos e na prevenção de erros comuns de programação.

    \subsection*{\emph{Static Single Assignment} (SSA)}

    A análise de tempo de vida (\emph{lifetime analysis}) em compiladores e \emph{garbage collectors} refere-se à
    determinação do intervalo de tempo durante o qual uma variável ou objeto permanece ativo e acessível no programa.
    Em relação à \emph{Static Single Assignment} (SSA), há uma conexão interessante. A SSA é uma forma especial de representação
    intermediária usada em compiladores onde cada variável recebe uma única atribuição, o que simplifica muitas análises e otimizações.
    Na SSA as variáveis são mais fáceis de serem rastreadas, e as informações sobre o tempo de vida podem ser derivadas mais claramente.
    Por conta disso, a SSA facilita algumas análises, incluindo a análise de tempo de vida, dado que a representação clara das
    atribuições torna mais fácil seguir a ``vida" de uma variável ao longo do programa.

    É importante notar que a análise de tempo de vida pode impactar a forma SSA, pois a identificação de quando uma variável não
    é mais usada pode levar à eliminação de atribuições desnecessárias em SSA.

    Em resumo, a análise de tempo de vida é fundamental para otimizações e \emph{garbage collectors}, e a representação em SSA pode
    facilitar a realização dessas análises, tornando-as mais eficientes e compreensíveis. A combinação dessas técnicas contribui
    para melhorar o desempenho e a eficiência dos programas compilados.

    \subsection*{\emph{Control Flow Graph} (CFG)}

    O fluxo de grafo de controle (\emph{Control Flow Graph} ou CFG) é uma representação fundamental em linguagens
    de programação usada principalmente na análise e otimização de compiladores. Ele não é específico para uma
    linguagem de programação em particular, mas é uma ferramenta universal aplicável a praticamente qualquer linguagem.
    O CFG é um conceito de nível de compilador, e não algo que os programadores escrevem diretamente no código-fonte.

    No entanto, para esclarecer o contexto, aqui estão alguns exemplos de linguagens de programação onde os
    compiladores ou ferramentas de análise estática frequentemente empregam CFGs:

    \begin{enumerate}
        \item \textbf{Java:} Compiladores de Java, como o Java HotSpot, também usam CFG para otimizar bytecode Java antes
        de ser executado na máquina virtual Java (JVM).

        \item \textbf{Python:} Ferramentas de análise estática para Python, como PyLint ou PyChecker, podem usar CFGs para entender
        o fluxo de controle do programa e identificar problemas como variáveis não utilizadas ou inacessíveis.

        \item \textbf{Go:} O compilador de Go usa CFGs para otimização e para garantir características como a ausência de
        \emph{deadlocks}.
    \end{enumerate}

    Em resumo, o fluxo de grafo de controle é uma abordagem usada em compiladores e ferramentas de análise de código
    para diversas linguagens de programação, ajudando na otimização e na garantia da corretude do código.

    \section*{\emph{Garbage Collector} da JVM}

    A \emph{Young Generation}, ou geração jovem, é uma parte essencial do sistema de gerenciamento de memória do Java.
    Compreendendo as áreas de \emph{Eden Space} e duas \emph{Survivor Spaces} (S0 e S1), a \emph{Young Generation} é o local onde os objetos
    recém-criados são alocados. Durante a execução do programa, o coletor de lixo atua de forma eficiente na \emph{Young Generation},
    implementando o algoritmo \emph{Garbage Collection} para identificar e remover objetos não referenciados. Este processo é
    fundamental para garantir uma utilização eficaz da memória e promover a rápida liberação de recursos de objetos temporários.
    Os objetos que sobrevivem a várias coletas na \emph{Young Generation} são promovidos para a \emph{Old Generation}, onde sua vida útil é
    estendida, contribuindo para um gerenciamento eficiente dos recursos de memória no ambiente Java.

    A \emph{Old Generation}, ou geração mais antiga, é uma parte crucial do gerenciamento de memória em Java. Nessa área,
    objetos com vida útil mais longa, que sobreviveram a várias coletas na \emph{Young Generation}, são alocados.
    Diferentemente da \emph{Young Generation}, a \emph{Old Generation} é mais resistente a mudanças frequentes e destina-se a
    objetos que persistem por mais tempo durante a execução do programa. O coletor de lixo atua na \emph{Old Generation}
    para identificar e remover objetos não referenciados, evitando vazamentos de memória e garantindo a eficiência
    no uso dos recursos do sistema. O equilíbrio entre a \emph{Young} e \emph{Old Generation} é fundamental para o desempenho e a
    estabilidade de aplicativos Java de longa duração.

    O \emph{Metaspace} é uma área especializada de armazenamento de metadados em Java, destinada a substituir o
    antigo espaço de permanente (\emph{PermGen}). Ao contrário das gerações de memória, o \emph{Metaspace} não está limitado por
    um tamanho fixo, pois pode crescer dinamicamente conforme necessário. Ele gerencia informações sobre classes,
    métodos e estruturas de dados relacionadas à execução do programa. A utilização do Metaspace é essencial para
    evitar problemas de falta de espaço para o armazenamento de metadados, oferecendo flexibilidade e adaptabilidade
    ao ambiente de execução Java.

    O congelamento temporário do programa durante a execução do coletor de lixo (GC) é uma característica inerente
    ao processo de gerenciamento de memória em Java. Durante esse período, conhecido como ``pausa de GC", a
    execução do programa é interrompida para permitir que o coletor de lixo identifique e remova objetos não
    referenciados, liberando assim recursos de memória. Embora essa pausa seja necessária para garantir a integridade
    e eficiência do gerenciamento de memória, é importante otimizar a configuração do GC e ajustar estratégias de
    coleta para minimizar o impacto no desempenho global do programa.

    \section*{\emph{Gargage Collector} da linguagem GO}

    O algoritmo \textit{Tri-color Mark and Sweep} é um método de coleta de lixo utilizado em várias linguagens de programação,
    incluindo Go. Este algoritmo é conhecido por sua eficiência e pela capacidade de minimizar as pausas no programa durante a coleta de lixo.

    O algoritmo divide os objetos em três ``cores": branco, cinza e preto. Cada cor representa um estado diferente
    no ciclo de vida da coleta de lixo.

    \subsection*{Fase de Marcação (\emph{Mark})}

    \begin{enumerate}
        \item \textbf{Inicialização:} Todos os objetos são marcados como brancos.
        \item \textbf{Raízes:} As raízes (objetos diretamente acessíveis, como variáveis globais e pilha de chamadas) são marcadas como cinzas.
        \item \textbf{Processamento dos Cinzas:} Cada objeto cinza tem seus filhos (referências) marcados como cinzas
        e ele mesmo é marcado como preto. Este processo continua até que não haja mais objetos cinzas.
    \end{enumerate}

    \subsubsection*{Fase de Varredura (\emph{Sweep})}
    \begin{enumerate}
        \item \textbf{Identificação de Objetos Brancos}: Após a marcação, os objetos que permanecem brancos
        são aqueles que não são acessíveis pelo programa.
        \item \textbf{Liberação de Memória}: Estes objetos brancos são então coletados, liberando a memória que ocupavam.
    \end{enumerate}

    As vantagens desse algoritmo são a minimização de pausas. O algoritmo permite a coleta de lixo incremental, reduzindo as pausas no programa.
    Além da eficiência, apenas os objetos inacessíveis são coletados, otimizando o uso de memória.

    O algoritmo \textit{Tri-color Mark and Sweep} é uma abordagem eficiente para a coleta de lixo em Go,
    equilibrando a necessidade de gerenciamento de memória com a minimização de interrupções no programa.

    \section*{Relação entre CFGs e \emph{Garbage Collectors}}

    \subsubsection*{Grafo de Fluxo de Controle (CFG)}

    \begin{itemize}
        \item \textbf{Propósito:} O CFG é uma representação abstrata usada para analisar o fluxo de execução de um programa.
        Ele mostra os caminhos possíveis que o controle pode seguir durante a execução do programa.

        \item \textbf{Uso em Compiladores:} Os CFGs são utilizados em compiladores para otimizações e análises estáticas,
        como a eliminação de código morto, análise de dependência de dados, e outras otimizações de tempo de compilação.

        \item \textbf{Natureza:} É uma representação estática do código, mostrando estruturas como loops, condições e saltos.
    \end{itemize}

    \subsubsection*{Coletor de Lixo da linguagem GO}

    \begin{itemize}
        \item \textbf{Propósito:} Este é um algoritmo de coleta de lixo usado para gerenciar a memória dinâmica.
        Ele identifica quais objetos na memória heap não são mais acessíveis pelo programa e,
        portanto, podem ser coletados (liberados).
        \item \textbf{Fases do Algoritmo:}
            \subitem \textbf{Marcação (\emph{Mark}):} Os objetos são marcados como ``acessíveis" ou ``não acessíveis"
            com base em se eles são referenciados pelo programa.
            \subitem \textbf{Varredura (\emph{Sweep}):} Os objetos não acessíveis são identificados e a memória que ocupam é liberada.
        \item \textbf{Natureza:}  É um processo dinâmico que acontece em tempo de execução e está mais
        relacionado à gestão de recursos de memória do que ao fluxo de controle do programa.

    \end{itemize}

    \subsubsection*{Diferenças Fundamentais}

    \begin{itemize}
        \item \textbf{Objetivo:} CFG é sobre o fluxo de controle (lógica de execução do programa), enquanto o
        Tri-color Mark and Sweep é sobre gerenciamento de memória (coleta de lixo).

        \item \textbf{Aplicação:} CFG é uma ferramenta de análise de código usada em tempo de compilação,
        enquanto o algoritmo de coleta de lixo é um processo de runtime que lida com a alocação e desalocação de memória.

        \item \textbf{Representação:} CFG é uma representação gráfica do código, enquanto o Tri-color Mark
        and Sweep não possui uma representação gráfica associada ao fluxo de controle do código.

    \end{itemize}

    \section*{Conclusão}

    Este resumo examinou o impacto e a eficácia dos coletores de lixo (\emph{garbage collectors}) na programação.
    Observamos que, embora coletores de lixo possam significativamente melhorar a produtividade do programador
    ao automatizar o gerenciamento de memória, eles também introduzem um \emph{overhead} que, em alguns casos, pode
    comprometer o desempenho em comparação com a otimização manual feita por programadores experientes.

    Por um lado, os coletores de lixo liberam os programadores da complexa e propensa a erros tarefa de
    gerenciar manualmente a memória, permitindo-lhes focar na lógica do aplicativo e na resolução de problemas
    mais diretamente relacionados aos objetivos do projeto. Essa automação pode levar a um desenvolvimento mais
    rápido e menos propenso a bugs relacionados à memória, como vazamentos ou corrupção de dados.

    Por outro lado, o \emph{overhead} introduzido pelos coletores de lixo, particularmente em termos de desempenho e uso
    de memória, pode ser uma desvantagem em certos cenários, especialmente onde a eficiência de recursos é crítica.
    Programas otimizados manualmente por desenvolvedores competentes podem superar os gerenciados automaticamente em
    termos de desempenho bruto e eficiência de memória.

    No entanto, em projetos de grande escala, o custo de execução dos coletores de lixo se torna proporcionalmente
    menos significativo. À medida que a complexidade do projeto aumenta, a sobrecarga administrativa de gerenciar
    manualmente a memória pode se tornar impraticável, fazendo com que os benefícios de um coletor de lixo superem
    seus custos. Em tais cenários, o uso de um coletor de lixo pode ser a escolha mais sensata, fornecendo uma solução
    escalável e menos propensa a erros para o gerenciamento de memória.

    Em conclusão, enquanto os coletores de lixo podem introduzir alguma sobrecarga, seu impacto tende a diminuir
    em projetos maiores, onde as vantagens de produtividade e segurança proporcionadas por eles se tornam mais evidentes.
    A escolha entre coleta de lixo automática e gerenciamento de memória manual deve ser feita considerando o contexto
    específico e os requisitos de desempenho de cada projeto.

\end{document}
